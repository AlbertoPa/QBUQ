\documentclass[a4paper,12pt,titlepage]{article}

%% Use the option review to obtain double line spacing
% \documentclass[authoryear,preprint,review,12pt]{elsarticle}

%% Use the options 1p,twocolumn; 3p; 3p,twocolumn; 5p; or 5p,twocolumn
%% for a journal layout:
%% \documentclass[final,authoryear,1p,times]{elsarticle}
%% \documentclass[final,authoryear,1p,times,twocolumn]{elsarticle}
%% \documentclass[final,authoryear,3p,times]{elsarticle}
%% \documentclass[final,authoryear,3p,times,twocolumn]{elsarticle}
%% \documentclass[final,authoryear,5p,times]{elsarticle} 
%  \documentclass[final,authoryear,5p,times,twocolumn]{elsarticle}

%% if you use PostScript figures in your article
%% use the graphics package for simple commands
%% \usepackage{graphics}
%% or use the graphicx package for more complicated commands
%% \usepackage{graphicx}
%% or use the epsfig package if you prefer to use the old commands
%% \usepackage{epsfig}

%\usepackage{subfigure}
%\usepackage{epstopdf}

%% The amssymb package provides various useful mathematical symbols
\usepackage{amsmath,amssymb}
%% The amsthm package provides extended theorem environments
%% \usepackage{amsthm}

%% The lineno packages adds line numbers. Start line numbering with
%% \begin{linenumbers}, end it with \end{linenumbers}. Or switch it on
%% for the whole article with \linenumbers after \end{frontmatter}.
%% \usepackage{lineno}

%% natbib.sty is loaded by default. However, natbib options can be
%% provided with \biboptions{...} command. Following options are
%% valid:

%%   round  -  round parentheses are used (default)
%%   square -  square brackets are used   [option]
%%   curly  -  curly braces are used      {option}
%%   angle  -  angle brackets are used    <option>
%%   semicolon  -  multiple citations separated by semi-colon (default)
%%   colon  - same as semicolon, an earlier confusion
%%   comma  -  separated by comma
%%   authoryear - selects author-year citations (default)
%%   numbers-  selects numerical citations
%%   super  -  numerical citations as superscripts
%%   sort   -  sorts multiple citations according to order in ref. list
%%   sort&compress   -  like sort, but also compresses numerical citations
%%   compress - compresses without sorting
%%   longnamesfirst  -  makes first citation full author list
%%
%% \biboptions{longnamesfirst,comma}

% \biboptions{}

%\journal{Nuclear Physics B}

%\usepackage{endfloat}

\usepackage{parskip}
\setcounter{secnumdepth}{5}
\setcounter{tocdepth}{5}

\newcommand{\xbf}{\mathbf{x}}
\newcommand{\xibf}{\boldsymbol{\xi}}
\newcommand{\Psibf}{\boldsymbol{\Psi}}
\newcommand{\samplespace}{\Omega}
\newcommand{\probspace}{\mathcal{P}}
\newcommand{\sigmaalgebra}{\mathfrak{F}}

\newcommand{\drm}{\textrm{d}}
\newcommand{\Mrm}{\textrm{M}}
\newcommand{\Nrm}{\textrm{N}}
\newcommand{\Qrm}{\textrm{Q}}



%\begin{frontmatter}

%% Title, authors and addresses
%\title{An Example to Run Quadrature-Based Uncertainty Quantification Package for 
%MFIX}

%\author{X.~Hu, A.~Passalacqua, F.~O.~Fox}
%\ead{xhu@iastate.edu}
%\address[Ad1]{Department of Chemical and Biological Engineering, Iowa State
%University, 2114 Sweeney Hall, Ames, IA 50011-2230, USA}
%\author{A. Passalacqua}
%\ead{albertop@iastate.edu}
%\address[Ad2]{Department of Mechanical  Engineering, Iowa State
%University, 2074 Black Engineering Building, Ames, IA 50011-2161, USA}
%\cortext[cor1]{Corresponding author}
%\author[Ad3]{Prakash Vedula}
%\address[Ad3]{School of Aerospace and Mechanical Engineering, University of
%Oklahoma, Norman, OK 73019-0601, USA}
%\ead{pvedula@ou.edu}
%\author{R.~O.\ Fox}
%\ead{rofox@iastate.edu}

%\begin{abstract}
%% abstract here
%\end{abstract}

%\begin{keyword}
%% keywords here, in the form: keyword \sep keyword

%% MSC codes here, in the form: \MSC code \sep code
%% or \MSC[2008] code \sep code (2000 is the default)

%\end{keyword}

%\end{frontmatter}

% \linenumbers

\begin{document}

\begin{titlepage}
\title{An Example to Run Quadrature-Based Uncertainty Quantification Package 
 for MFIX}

\author{X. Hu \\ 
        xhu@iastate.edu \\
        Department of Chemical and Biological Engineering \\
        Iowa State University \\
        Ames, IA, USA \\
        \and
        A. Passalacqua \\
        albertop@iastate.edu \\
        Department of Mechanical  Engineering \\
        Iowa State University \\
        Ames, IA, USA \\
        \and
        R. O. Fox \\
        rofox@iastate.edu \\
        Department of Chemical and Biological Engineering \\
        Iowa State University \\
        Ames, IA, USA}
\date{December 3rd, 2013}
\maketitle
\end{titlepage}

%\tableofcontents
%% main text
\section{Introduction}
\label{sec:Introduction}
This document provides an example of using the non-intrusive quadrature-based 
uncertainty quantification (QBUQ) package implemented in MFIX. The package is 
developed in two parts: pre-processing and post-processing modules. In the
pre-processing module, the space of the uncertain input parameters is sampled
first, quadrature weights and nodes are obtained, and corresponding MFIX input
file mfix.dat for each sample is generated. Then a set of MFIX simulations
can be performed to obtain quadrature abscissas. In the post-processing module,
moments of the system response can be estimated. Then low order statistics, 
including mean, variance, skewness, and kurtosis can be obtained, and the
probability distribution function (PDF) of the system response at the specific
location can be reconstructed. In the following parts, MFIX tutorial case 
fluidBed1 with uncertain particle size is used as an example case to illustrate 
the QBUQ procedure. 

\section{An example to run QBUQ package}
\label{sec:ExampleQBUQ}
The QBUQ modules are written in python3. The example shown here is using
python3 (version 3.2.3) with following packages:\ numpy (version 1.7.1), scipy 
(version 0.12.0), sympy (version 0.7.2). Simulations are run using MFIX-2013-1, 
and bash shell is used for shell scripts.

MFIX tutorial case fulidBed1 is used as an example case in this section. It is 
a 2D bubbling fluidized bed reactor with a central jet. The uncertain parameter
in this example is particle diameter, distributed uniformly from 
$300\ \mu\textrm{m}$ to $500\ \mu\textrm{m}$. To use the QBUQ package, create a 
new run directory, and put the following directory and files in your run
directory:\ directory ``qbuq\_package'', python3 files ``pre\_qbuq.py'' and 
``post\_qbuq.py'', and bash script files ``run\_serial'' and ``extract\_data''. 
You also need to provide a basic mfix.dat file in the run directory to use 
for generating input mfix.dat file for each sample.

\subsection{Pre-processing module}
\label{sec:Pre-processing}
The pre-processing module samples the space of uncertain parameters (particle
diameter here), generates quadrature weights and nodes, and create corresponding
mfix input files. Python3 file ``pre\_qbuq.py'' is the main script, and 
corresponding functions are in subpackage ``pre\_processing'' under package 
``qbuq\_package''. To run the module, first go to your \emph{run directory} in 
command window, then type: 

\textbf{python3 pre\_qbuq.py}

The program asks you to choose number of uncertain parameters. Only univariate 
and bivariate problems are implemented for now. For our case, the only uncertain
parameter is particle diameter, so type \textbf{1} to select ``1 -- Univariate 
problem''.

Then the program asks if the distribution of the variable is known. Type 
\textbf{y} for yes here since the particle diameter in our example is uniformly
distributed. In the next prompt, type \textbf{1} to choose ``Uniform 
distribution''.

Next the program asks how many samples you need to generate. Here we use 
\textbf{10} samples in the example. Then you are asked for the range of your 
uniformly distributed uncertain parameter. Remember, in the basic MFIX input 
file mfix.dat, the units system we use is ``\emph{cgs}''. Therefore, our range 
of particle diameter is $0.03\ \textrm{cm}$ to $0.05\ \textrm{cm}$. Type 
\textbf{0.03} as the minimum value and \textbf{0.05} as the maximum value.

So far the first step, sample the space of uncertain parameter is finished. The
program says a text file containing quadrature weights and nodes are generated 
successfully. The name of this file is ``quadrature\_weights\_nodes.txt'', in 
which the first row shows quadrature weights, and the second row shows 
quadrature nodes.

Then the program starts to generate MFIX input files for each sample. First a 
case sensitive head of the run\_name is asked. Then the run\_name for each 
mfix.dat file will be this head followed by a number. In our case, type 
\textbf{Test}, and with increasing particle size, the run\_name will be Test0 to 
Test9.

Next the program asks how many keywords in mfix.dat file use the value of the 
uncertain parameter. Please count carefully to include all keywords related to
the uncertain parameter. In our case, only one keyword ``\emph{D\_p0}'' is 
related to particle diameter. Type \textbf{1} here, and type \textbf{D\_p0} for 
the next prompt. \emph{\underline{Note:}} The keywords typed here should be 
exactly the same with those in the basic mfix.dat file because these keywords 
will be searched and their values will be replaced by the quadrature nodes. If 
the keywords cannot match, the values may not be replaced.

Then the program searches keywords ``\emph{run\_name}'' and  ``\emph{D\_p0}'' in 
the basic mfix.dat file and replaces their values with the new run\_name and 
quadrature node value, respectively. The MFIX input file mfix.dat generated for 
each sample is stored in separated directories named by their run\_name. A text
file named ``\emph{list\_of\_cases.txt}'' is also generated, which contains 
run\_name of each sample in the order of increasing quadrature nodes so that in
post-processing part quadrature weights and nodes are corresponded. 

Now the pre-processing part of QBUQ procedure is finished. Program says MFIX 
input files are generated successfully. In the run directory, you now have 2 
more text files, ``quadrature\_weights\_nodes.txt'' and ``list\_of\_cases.txt'', 
and 10 more folders named from Test0 to Test9, each one containing a MFIX input 
file mfix.dat. The only difference of these input files is the value of 
\emph{D\_p0}. Next is how to use the shell script ``run\_serial'' to run these 
simulations in serial.

\subsection{Running simulations in serial}
\label{sec:RunSerial}
A bash script ``run\_serial'' is written to run generated MFIX simulations in 
serial. In this example, when compile MFIX, serial mode of execution for each 
sample is selected. To run MFIX in DMP or SMP mode, please change commands
in this file accordingly. If you want to submit your jobs on a cluster through
a batch queue system, please consult with your system administrator.

To use the script ``run\_serial'', in the run directory, type:

\textbf{bash run\_serial MFIX\_model\_directory}

Here the ``MFIX\_model\_directory'' is the path of your MFIX model directory, 
such as ``/home/mfix/model''.

Then the program prompts the initial compilation of MFIX. The options used in 
this initial compilation will be used for all the simulations generated 
in~\ref{sec:Pre-processing}. The options used in this example are following:

\begin{itemize}
 \item Mode of execution: [1] Serial
 \item Level of optimization: [3] Level 3 (most aggressive)
 \item Option to re-compile source files in run directory: [1] Do not force 
 re-compilation
 \item Compiler selection: [1] GNU (gfortran) version 4.3 and above
\end{itemize}

After the initial compilation, the same options are used repeatedly for all 
simulations. From Test0 to Test9, simulations are compiled and run in the order
listed in text file ``list\_of\_cases.txt''. The screen outputs of compilation 
and run for each simulation are redirected to log files ``compile.log'' and
``run.log'', respectively. Once all the simulations are completed, time-averaged 
quantities can be extracted, and the post-processing module of QBUQ can be used.

\subsection{Post-processing module}
\label{sec:Post-processing}

The post-processing module first extracts time-averaged quantities with a bash
shell script ``extract\_data'' by calling MFIX post-processing program 
``post\_mfix''. Then time-averaged results can be used for QBUQ post-processing 
module ``post\_qbuq.py''. The module can perform 2 tasks:\ estimate moments and 
low order statistics (mean, variance, skewness, and kurtosis), and reconstruct 
the PDF of system response. Functions used in this module are in subpackage 
``post\_processing'' under package ``qbuq\_package''.

\subsubsection{Extracting time-averaged data}
\label{sec:ExtractData}

To extract time-averaged quantities, a basic text file need to be provided in 
the run directory to use as the input file for MFIX post-processing program
``post\_mfix''. Then the bash shell script ``extract\_data'' will replace the 
run\_name automatically and extract the data for each simulation. In this
example, 2 basic text files are provided in the run directory:\ 
``post\_all.txt'' extracts time-averaged quantities on the whole computational
domain which will be used for estimation of moments and low order statistics, 
and ``post\_point.txt'' extracts data at one specific location which will be 
used for reconstruction of the PDF. The following parts show what conditions are 
used in these two post\_mfix input files for MFIX post-processing program . 
\emph{\underline{Note:}} When name the file stored the extracted data, a .txt 
file extension must be added. Only .txt files can be read by the post\_qbuq 
module.

\paragraph*{post\_all.txt}

\begin{itemize}
 \item RUN\_NAME to post\_procss $>$ Test (\emph{\underline{Note:}} This name 
 will be replaced automatically with the run\_name of each simulation.)
 \item Enter menu selection $>$ 1 - Examine/print data
 \item Write output using user-supplied precision? (T/F) $>$ f
 \item Time:(0.000, 0.000) $>$ 0, 2
 \item Time average? (N) $>$ y
 \item Variable:\ (EP\_g) $>$ EP\_g
 \item I range:\ (1, 1) $>$ 1, 9
 \item Average or sum over I? (N) $>$ n
 \item J range:\ (1, 1) $>$ 1, 102
 \item Average or sum over J? (N)  $>$ n
 \item K range:\ (1, 1) $>$ 1, 1
 \item File:\ (*) $>$ EP\_g\_all.txt (\emph{\underline{Note:}} Must use .txt 
 file here. The post\_qbuq module only reads .txt files.)
 \item Time:(0.000, 0.000) $>$ -1
 \item Enter menu selection $>$ 0 - Exit POST\_MFIX
\end{itemize}

Only gas volume fraction ``\emph{EP\_g}'' is extracted on the whole 
computational domain as an example. Other time-averaged quantities can be 
extracted by changing or adding conditions in the file accordingly. 

\paragraph*{post\_point.txt}

\begin{itemize}
 \item RUN\_NAME to post\_procss $>$ Test
 \item Enter menu selection $>$ 1 - Examine/print data
 \item Write output using user-supplied precision? (T/F) $>$ f
 \item Time:(0.000, 0.000) $>$ 0, 2
 \item Time average? (N) $>$ y
 \item Variable:\ (EP\_g) $>$ EP\_g
 \item I range:\ (1, 1) $>$ 7, 7
 \item J range:\ (1, 1) $>$ 51, 51
 \item K range:\ (1, 1) $>$ 1, 1
 \item File:\ (*) $>$ EP\_g.txt (\emph{\underline{Note:}} Must use .txt 
 file here. The post\_qbuq module only reads .txt files.)
 \item Time:(0.000, 0.000) $>$ 0, 2
 \item Time average? (N) $>$ y
 \item Variable:\ (EP\_g) $>$ P\_g
 \item I range:\ (1, 1) $>$ 7, 7
 \item J range:\ (1, 1) $>$ 51, 51
 \item K range:\ (1, 1) $>$ 1, 1
 \item File:\ (*) $>$ P\_g.txt
 \item Time:(0.000, 0.000) $>$ 0, 2
 \item Time average? (N) $>$ y
 \item Variable:\ (EP\_g) $>$ V\_s
 \item I range:\ (1, 1) $>$ 7, 7
 \item J range:\ (1, 1) $>$ 51, 51
 \item K range:\ (1, 1) $>$ 1, 1
 \item File:\ (*) $>$ V\_s.txt 
 \item Time:(0.000, 0.000) $>$ -1
 \item Enter menu selection $>$ 0 - Exit POST\_MFIX
\end{itemize}

Gas volume fraction ``\emph{EP\_g}'', gas pressure ``\emph{P\_g}'' and vertical
solid velocity ``\emph{V\_s}'' are extracted at a specific location. Conditions
in the file can be changed or added accordingly.

To use the bash shell script ``extract\_data'' to extract time-averaged data for
all simulations, in the run directory, type:

\textbf{bash extract\_data post\_file post\_mfix\_directory}

Here the ``post\_file'' is the file used as the input for MFIX post-processing
program ``post\_mfix''. In this example, it can be either ``post\_all.txt'' or
``post\_point.txt''. The ``post\_mfix\_directory'' is the path of your MFIX
post\_mfix directory, such as ``/home/mfix/post\_mfix''. We run the script
twice using both ``post\_all.txt'' and ``post\_point.txt'' to extract gas 
volume fraction ``\emph{EP\_g}'' on the whole computational domain, and gas
volume fraction ``\emph{EP\_g}'', gas pressure ``\emph{P\_g}'' and vertical
solid velocity ``\emph{V\_s}'' at a specific location for all 10 simulations. 
Once the process is completed, these data can be used for post\_qbuq module.

\subsubsection{Estimation of moments and low order statistics}
\label{sec:LowOrder}

Two tasks can be performed by the QBUQ post-processing module, estimation of 
moments and low order statistics, and reconstruction of the PDF of system
response. Here first shows estimation of moments and low order statistics. 
To run the module, in the run directory, type:

\textbf{python3 post\_qbuq.py}

The program asks you to select your job first. Type \textbf{1} to to choose
``1 -- Calculate low order statistics''.

Then the program asks for the file name of the quantity of interest. In this 
example, it is the file name used in file ``post\_all.txt'', ``EP\_g\_all.txt''.
So type \textbf{EP\_g\_all.txt} to continue. Next the program asks how many 
samples we have. Type \textbf{10} for 10 samples in this example.

Then we are asked for the highest order of moments. In order to estimate low
order statistics including mean, variance, skewness, and kurtosis, at least 
fourth order moment is needed. So type \textbf{4} here to continue.

Next the program uses the time-averaged data of each simulation extracted 
in~\ref{sec:ExtractData} as quadrature abscissas and the quadrature weights 
generated in~\ref{sec:Pre-processing} to estimate up to fourth order moments of 
gas volume fraction ``\emph{EP\_g}'' on the whole computational domain and 
calculate mean, variance, skewness, and kurtosis as well. Runtime warning maybe
displayed here because when calculate skewness and kurtosis, the standard
deviation as the denominator may become 0 at those locations, which results in 
``NaN'' or ``Inf'' for skewness and kurtosis. These warning do not effect the 
results at other locations. 

Once the program is completed, 5 more text files are
generated in the run directory:\ ``moments.txt'', ``mean.txt'', 
``variance.txt'', ``skewness.txt'', and ``kurtosis.txt''. Make sure these files 
are stored and renamed correctly before using the post\_qbuq module again. 
These files will be rewritten when the module is used for other quantities of 
interest.

\subsubsection{Reconstruction of the PDF}
\label{sec:Reconstruction}

The other task the QBUQ post-processing module can perform, reconstruction of 
the PDF of system response, is shown here. PDF of three quantities, gas
volume fraction ``\emph{EP\_g}'', gas pressure ``\emph{P\_g}'' and vertical
solid velocity ``\emph{V\_s}'', at a specific location are reconstructed using
extended quadrature method of moments 
(EQMOM)\cite{Chalons2010,YuanLaurentFox2011}. 
The following parts describe the reconstruction of the PDF of these three 
quantities one by one.

\paragraph{Gas volume fraction ``\emph{EP\_g}''}\mbox{}\\
\label{sec:ReconEPg}

Type the following command to run the QBUQ post-processing module.

\textbf{python3 post\_qbuq.py}

Type \textbf{2} to select ``2 -- Reconstruction the probability distribution
function''. Then enter \textbf{1} to choose $\beta$-EQMOM for \emph{EP\_g}.

Then the program asks for the number of nodes to use. Maximum 5 nodes can be 
used. Type \textbf{4} for our example. Next the program asks for 4 ratios of 
minimum weights to maximum weights (rmin), which are parameters used in adaptive
Wheeler algorithm\cite{YuanFox2011}. For $rmin(0)$, type \textbf{0}; type 
$\mathbf{1e-12}$ for $rmin(1)$; for $rmin(2)$, enter $\mathbf{1e-8}$; enter 
$\mathbf{1e-4}$ for $rmin(3)$. In the next prompt, the program asks to enter the
minimum distance between distinct nodes (eabs), which is also a parameter used 
in adaptive Wheeler algorithm\cite{YuanFox2011}. Enter $\mathbf{1e-10}$ here.

Next type \textbf{10} for we have in total 10 simulations in this example. Then
for the file name of quantity of interest, type \textbf{EP\_g.txt}.

In $\beta$-EQMOM, a bounded interval is needed. In the post\_qbuq module, the 
bounded interval can either use the minimum to maximum value of the data or be
set up manually. In this example, type \textbf{1} to use minimum to maximum
value of the data as the interval.

Then the program uses $\beta$-EQMOM to reconstruct the PDF of gas volume 
fraction. Once it is completed, information of $\sigma$, weights, nodes on 
$[0,1]$, and nodes on bounded interval is shown on screen. Three more text files
are generated in the run directory. File ``betaEQMOM\_sigma.txt'' keeps the 
value of $\sigma$. In file ``betaEQMOM\_weights\_nodes.txt'', first row shows
weights of each EQMOM node, second row gives value of each EQMOM node on 
$[0,1]$, and the third row shows the value of each EQMOM node on the bounded
interval. File ``data\_set\_for\_betaEQMOM.txt'' contains 10 values used to 
reconstruct the gas volume fraction, each of which is the value of gas volume
fraction of that simulation at the specific location.

For your reference, Table~\ref{tab:ReconEPg} gives the results shown on the 
author's screen.

\begin{table}[htp]
 \centering
 \begin{tabular}{l|ll} \hline
  sigma       & \multicolumn{2}{l}{0.00523497997459} \\ \hline
  weights     & 0.46426887803    & 0.284369847782    \\ 
              & 0.217899463915   & 0.033461810273    \\ \hline
  nodes       & 0.00956478087672 & 0.0894838372351   \\
  on $[0, 1]$ & 0.629186940922   & 0.999750406862    \\ \hline
  nodes       & 0.415734483248   & 0.425965720843    \\ 
  on $[a, b]$ & 0.495058512177   & 0.542498047086    \\ \hline
 \end{tabular}
 \caption{Reconstruction results of gas volume fraction using $\beta$-EQMOM.}
 \label{tab:ReconEPg}
\end{table}

\paragraph{Gas pressure ``\emph{P\_g}''}\mbox{}\\
\label{sec:ReconPg}

The procedure of using post\_qbuq module to reconstruct the PDF of gas pressure
\emph{P\_g} at the location is similar to the procedure used 
in~\ref{sec:ReconEPg} except that some different options are selected.
Conditions used in the module to reconstruct the PDF of gas pressure are as
follows. 

\begin{itemize}
 \item Type \textbf{python3 post\_qbuq.py} to run the module.
 \item Type \textbf{2} to select reconstruction of the PDF.
 \item Type \textbf{2} to use $\gamma$-EQMOM to reconstruct the PDF of gas 
 pressure.
 \item Enter \textbf{5} as the number of EQMOM nodes.
 \item For the parameters rmin in adaptive Wheeler algorithm, the first four are 
 the same with~\ref{sec:ReconEPg}. Type $\mathbf{1e-4}$ for the fifth one 
 $rmin[4]$.
 \item Use $\mathbf{1e-10}$ again as eabs in adaptive Wheeler algorithm.
 \item Enter \textbf{10} for 10 samples.
 \item The file name of gas pressure in this example is \textbf{P\_g.txt}.
 \item Select \textbf{2} to set the lower bound for $\gamma$-EQMOM manually.
 \item Type \textbf{200} as the lower bound.
\end{itemize}

Once the reconstruction is completed, 3 more files are generated in the run
directory:\ ``gammaEQMOM\_sigma.txt'', ``gammaEQMOM\_weights\_nodes.txt'', and
``data\_set\_for\_gammaEQMOM.txt''. Table~\ref{tab:ReconPg} gives results on 
author's screen for your reference.

\begin{table}[htp]
 \centering
 \begin{tabular}{l|lll} \hline
  sigma            & \multicolumn{3}{l}{19.606992775}                 \\ \hline
  weights          & 0.177257589525 & 0.27693370017  & 0.181902294417 \\ 
                   & 0.150325247342 & 0.213581168547 &                \\ \hline
  nodes            & 283.874714748  & 1536.18411219  & 3175.10056231  \\
  on $[0, \infty)$ & 4271.74179277  & 6774.60334977  &                \\ \hline
  nodes            & 483.874714748  & 1736.18411219  & 3375.10056231  \\ 
  on $[a, \infty)$ & 4471.74179277  & 6974.60334977  &                \\ \hline
 \end{tabular}
 \caption{Reconstruction results of gas pressure using $\gamma$-EQMOM.}
 \label{tab:ReconPg}
\end{table}

\paragraph{Vertical solid velocity ``\emph{V\_s}''}\mbox{}\\
\label{sec:ReconVs}

The procedure of using post\_qbuq module to reconstruct the PDF of vertical
solid velocity \emph{V\_s} at the location is slightly different from the 
procedure used in~\ref{sec:ReconEPg} and~\ref{sec:ReconPg}, since $2$-node
Gaussian EQMOM is used\cite{Chalons2010}. The procedure is as follows.

\begin{itemize}
 \item Type \textbf{python3 post\_qbuq.py} to run the module.
 \item Type \textbf{2} to select reconstruction of the PDF.
 \item Type \textbf{3} to use $2$-node Gaussian EQMOM to reconstruct the PDF of 
 vertical solid velocity.
 \item Enter \textbf{10} for 10 samples.
 \item The file name of vertical solid velocity in this example is 
 \textbf{V\_s.txt}.
\end{itemize}

Three more files are generated in the run directory once the reconstruction is
completed:\ ``GaussianEQMOM\_sigma.txt'', ``GaussianEQMOM\_weights\_nodes.txt'', 
and ``data\_set\_for\_GaussianEQMOM.txt''. Table~\ref{tab:ReconVs} gives results 
on author's screen for your reference.

\begin{table}[htp]
 \centering
 \begin{tabular}{l|ll} \hline
  sigma   & \multicolumn{2}{l}{1.93623130711} \\ \hline
  weights & 0.694653795117 & 0.305346204883   \\ \hline
  nodes   & -5.1032997211  & 3.10647288573    \\ \hline
 \end{tabular}
 \caption{Reconstruction results of vertical solid velocity using $2$-node 
 Gaussian EQMOM.}
 \label{tab:ReconVs}
\end{table}

Now this document gives a complete example of using QBUQ package for a MFIX case
with one uncertain parameter.
%% The Appendices part is started with the command \appendix;
%% appendix sections are then done as normal sections
%% \appendix

%% \section{}
%% \label{}

%% References with bibTeX database:

\bibliographystyle{plain}
\bibliography{qbuq_example}

\end{document}
